% Created 2017-03-12 Sun 17:28
\documentclass[12pt]{amsart}
                        \usepackage[all]{pabmacros}
\usepackage[utf8]{inputenc}
\usepackage[T1]{fontenc}
\usepackage{fixltx2e}
\usepackage{graphicx}
\usepackage{longtable}
\usepackage{float}
\usepackage{wrapfig}
\usepackage[normalem]{ulem}
\usepackage{textcomp}
\usepackage{marvosym}
\usepackage[nointegrals]{wasysym}
\usepackage{latexsym}
\usepackage{amssymb}
\usepackage{amstext}
\usepackage{hyperref}
\tolerance=1000
\usepackage{amsmath}
\usepackage[bibstyle=alphabetic,citestyle=alphabetic,backend=bibtex]{biblatex}
\makeatletter
\def\blx@maxline{77}
\makeatother
\bibliography{refs}
\AtEveryBibitem{\clearfield{doi}}
\AtEveryBibitem{\clearfield{url}}
\AtEveryBibitem{\clearfield{issn}}
\renewcommand*{\bibfont}{\footnotesize}
\date{}
\renewcommand*{\bibfont}{\footnotesize}
\author{Paul Bryan}
\date{\today}
\title{C9. Research Opportunity and Performance Evidence (ROPE) - A statement of your Research Impact and contributions to the research field of this Proposal}
\hypersetup{
  pdfkeywords={},
  pdfsubject={},
  pdfcreator={Emacs 24.5.1 (Org mode 8.2.10)}}

\usepackage{enumitem}
\usepackage{setspace}
\usepackage[margin = 10mm]{geometry}

\doublespacing
\pagenumbering{gobble}
\pagestyle{empty}

\begin{document}
Dr. Bryan has been working on the connection between isoperimetric problems and curvature flows since his Ph.D. thesis. Using variational techniques, he discovered the underlying structure behind prior investigations \cite{MR1369140,MR1369139,MR1656553,Bryan} in papers published in top journals such as Crelle's journal and Calc. Var. PDE. This structure manifests itself through a viscosity differential inequality for the \emph{isoperimetric profile} and has inspired further research such as \cite{MR3544942, MR3570462}. One of the referees for Dr. Bryan's thesis described the exposition as "one of the clearest explanations of the isoperimetric profile". The work has featured in lecture notes by Asst. Prof. Robert Haslhofer, Dr. Mohammad Ivaki as well as in Prof. Richard Schoen's student Ben Stetler's honours thesis

The approach used, for instance Dr. Bryan's work with Prof. Ben Andrews \cite{MR2729306} extends the isoperimetric comparison techniques in \cite{MR1369140,MR1369139} to obtain a direct curvature bound for solutions of the normalised Ricci flow on the two-sphere. Such a result leads directly to a proof that solutions of the normalised Ricci flow smoothly converge to a constant curvature metric. Prior to this, the proof relied on a good deal of extra machinery, beginning with a proof for \emph{positively} curved metrics in \cite{MR954419} employing a Harnack inequality and entropy inequality and then later proven for general metrics in \cite{MR1094458} by employing an extension of the Harnack inequality and entropy inequality. The standout features of the isoperimetric comparison approach are the ability, via a comparison with a model solution, to obtain the direct curvature bounds and the isoperimetric constant. The two combined give extremely strong control of the flow, both geometrically and analytically; the latter since the isoperimetric constant equals the Sobolev constant.

Dr. Bryan extended these results to arbitrary surfaces \cite{Bryan} giving a unifying approach to the Ricci flow on surfaces obtaining convergence results for all closed surfaces by a single method. This has further inspired work such as \cite{2014arXiv1411.2672N} in proving isoperimetric inequalities via the viscosity method.

The paper \cite{MR2794630}, was described by the Mathematical Reviews of the AMS as an "\ldots{}article [that] presents another significant simplifying contribution". Dr. Bryan and Prof. Ben Andrews used the viscosity approach to again obtain strong control over the flow in the form of a curvature bound, leading directly to a proof of the Gage-Hamilton-Grayson theorem \cite{MR840401,MR906392}, that any embedded curve in the plane smoothly converges to a round circle under the normalised Curve Shortening Flow. The technique was inspired by the distance comparison principle in \cite{MR1656553} which has also been generalised to higher dimensions in \cite{MR2967056,MR3011290}. This work has featured in several sets of lecture notes on the Curve Shortening Flow such as those by Assistant Professor Robert Haslhofer at the University of Toronto and by Dr. Mohammad Ivaki at the Vienna University of Technology. Moreover, this paper strongly influenced the development in higher dimensions, which in turn heavily influenced the recent successes solving the Hsiang-Lawson conjecture \cite{MR3143888} for minimal tori in $\sphere^3$ and the Pinkall-Stirling conjecture for CMC tori in $\sphere^3$ \cite{2012arXiv1204.5007A}. The paper \cite{alpha_csf_dist_comp} in preparation further elucidates the underlying viscosity equation, and extends it to a greater class of flows.

As further testament to the importance of this approach, in \cite{MR2843240} yet another proof of the Gage-Hamilton-Grayson theorem was offered, obtaining a viscosity equation for the \emph{relative} isoperimetric profile for the Curve Shortening Flow. This required considerable analysis of the isoperimetric profile, particularly in a symmetric, convex model situation.

One striking feature of the comparison results described above, is that the model comparison used comes from an \emph{ancient} solution of the flow. Ancient solutions are solutions are important in that they model singularity formation of flows, a very active area of current research \cite{MR3020169,MR2993752}. Such solutions have been classified in certain situations in \cite{MR2669361, MR2971286} as either the trivial solutions or the model solutions used in the above comparison results.

Beginning with \cite{bryanlouie}, Dr. Bryan has worked with Dr. Mohammad Ivaki and Dr. Julian Scheuer developing a suite of results obtaining Harnack inequalities for hypersurface flows in Riemannian backgrounds and classifying ancient, convex solutions of hypersurface flows on the standard $n$-sphere \cite{BIS4,2016arXiv160401694B,2016arXiv160401694B,2015arXiv150802821B}. This work is quite recent, published over the last twelve months, but has already attracted citations. Of particular note, is that classification results are obtained for \emph{arbitrary, geometric, parabolic flows} in the sphere using a parabolic version of the Aleksandrov reflection technique. Results in such generality are entirely unknown outside this work, with the literature focusing on classic flows such as the Mean Curvature Flow and Gauss Curvature Flow; the most general results require flows with homogeneous speeds as well as convexity/concavity assumptions. The classification results described here are valid without any such restrictions, a fact that is sure to be of considerable interest.

Ancient solutions to evolution equations are important for several reasons. One major reason is that the self-similar solutions - the solitons - are used to model singularity formation \cite{MR1375255,MR1666878}. Closely related is the Harnack inequality for geometric flows \cite{MR1296393,MR1316556,MR1100812,MR1198607} which serves as an important tool in the analysis of singularity formation as well as solitons. In the series of papers \cite{bryanlouie,2016arXiv160401694B,2015arXiv150802821B,2015arXiv151203374B}, with Dr. Janelle Louie, Dr. Mohammad Ivaki and Dr. Julian Scheuer, Dr. Bryan has instigated a program of obtaining Li-Yau-Hamilton Harnack inequalities for hypersurface flows in Riemannian backgrounds. The non-linearities introduced by the background curvature become of significant importance, strongly restricting when such inequalities are possible. A rather significant fact is that such difficulties may be overcome at all - the computations involved, even in Euclidean space are notoriously difficult and the results described are the first known results in non-Euclidean background spaces.

In \cite{MR1296393} a method was introduced leading to significant simplifications in the Euclidean case, but the method does not readily generalise to other background spaces. Dr. Bryan's work with Dr. Mohammad Ivaki and Dr. Julian Scheuer on Harnack inequalities culminates in \cite{BIS4} where a new method is introduced to perform the necessary computations in \emph{arbitrary background spaces}. The idea is based on two incisive observations. The first is that given a homothetically scaling soliton, the change of parametrisation from scaling $F(x, t) = \lambda(t) F_0(x)$ to the standard parametrisation where $\partial_t F = -f(\mathcal{W})\nu$ introduces a gradient term. The second is that the Harnack quantity decreases most rapidly along the flow in the direction of this very same gradient. These observations provide an optimal framework in which to perform the computations, and also illuminate the connection between solitons and the Harnack inequality, previously only seen through considerable computations based on the soliton equation. Moreover, this work exhibits precisely the conditions required on the speed of the flow and the background curvature for which a Harnack inequality could hold. The level of generality of this paper is again not often seen in works on curvature flows.

The impact of this work, being very recent, is difficult to judge, but after giving several talks on the subject, the strong interest in the work is clear. The technique applies quite broadly and is appropriate for intrinsic flows such as the Ricci flow and scalar heat equations and will surely be developed on in the coming years.
% Emacs 24.5.1 (Org mode 8.2.10)
\end{document}
